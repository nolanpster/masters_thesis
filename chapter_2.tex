%
% Since this is a ``report'', the topmost level of hierarchy is
% ``Chapter'', not section as you may be used to. Chapters are
% enumerated starting from 1, so Sections are 1.1, Subsections are
% 1.1.1, subsubsections don't get numbers. (You can change that, if
% you want them to be called 1.1.1.1)
%
\chapter{First Chapter.}
This should ideally contain some text. 
\section{First Section.}
More Text.
\subsection[Alternative title for the Table of Contents]{First Subsection.}
Even more text, maybe a formula:
\begin{equation}
\sum_{i=1}^{n}i=\frac{n(n+1)}{2} % much easier than Microsoft Equation
                                 % Editor :)
\end{equation}
\subsubsection{First Sub-subsection.}
This is really deep down in the hierarchy. Maybe you shouldn't even
use sub\-subsections. It goes further down (paragraphs), but I don't
think you'll need that\footnote{By the way: notice that, although we
have doublespacing here, the footnotes are singlespaced. This is
intended and good. If you want to change that, try, but this is really
how it should be.}.

\section{Other thoughts.}
Okay, what else?
Let me quickly put a figure here, maybe a piece of pseudo code.
That way, you can see how this is done. It's a little painful, but
looks really cool. We will call it Figure~\ref{fig:source_algo1}. The
numbering is automatic---don't worry about it.



\noindent
And so on, and so on.

Please remember that you have to compile a document several times when
you did changes that affect figures, table of contents, bibliography,
etc. (This is always the case if you get the warning: ``LaTeX Warning:
Label(s) may have changed. Rerun to get cross-references right.'').

The recommended sequence is :

\texttt{latex foo.tex}

\texttt{bibtex foo}

\texttt{latex foo.tex}

\texttt{latex foo.tex}
